\documentstyle[12pt]{article}

\begin{document}

\title{Lessons on doing science from my father, Gerry Brown}
\author{C. Titus Brown\\
  Population Health and Reproduction\\
  School of Veterinary Medicine\\
  University of California, Davis\\
  ctbrown@ucdavis.edu}
\maketitle

Dr. Gerald E. Brown was a well known nuclear physicist and
astrophysicist who worked at Stony Brook University from 1968 until
his death in 2013. He was internationally active in physics research
from the late 1950s onwards, ran an active research group at Stony
Brook until 2009, and supervised nearly a hundred PhD students during
his life. He was also my father.

It's hard to write about someone who is owned, in part, by so many
people. I came along late in my father's life (he was 48 when I was
born), and so I didn't know him that well as an adult, but I did grow
up with a senior professor as a father, and so he had a tremendous
impact on not only my personal life but also my scientific career. Now
that I am a professor I recognize this even more clearly.

Gerry (as I called him) didn't spend much time directly teaching his
children about his work.  When I was invited to write something for
his memorial book, it was suggested that I write about what he had
taught me about being a scientist. I found myself stymied, because to
the best of my recollection we had never talked much about the
practice of science. When I mentioned this to my oldest brother, Hans,
we shared a good laugh because he had the same experience with our
father!

Most of what Gerry taught me was taught by example, then. Here are
some of the examples that I remember most clearly, and of which I'm
the most proud. While I don't know if either of my children will
become scientists, I can think of few better scientific legacies to
pass on to them from my father.

\section{Publishing work that is interesting (but perhaps not right) can make
for a fine career.}

My father was very proud of his publishing record, but not because he
was always (or even frequently) right. In fact, several people told me
that he was somewhat notorious for having a 1- in-10 'hit rate' ' he
would come up with many crazy ideas, of which only about 1 in 10 would
be worth pursuing. However, that 1 in 10 was enough for him to have a
long and successful career.  That this drove some people nuts was
merely an added bonus in his view.

Gerry was also fond of publishing controversial work. Several times he
told me he was proudest of the papers that caused the most discussion
and rebuttals, and he proudly noted that these papers often had the
largest numbers of citations, even when they had turned out to be
incorrect.

\section{The best collaborations are both personal and professional
friendships.}

The last twenty-five years of Gerry's life were dominated by a
collaboration with Hans Bethe on astrophysics, and they traveled to
Pasadena every January until the early 2000s to work at the California
Institute of Technology. During this month they lived in the same
apartment, worked together closely, and met with a wide range of
people on campus to explore scientific ideas; they also went on long
hikes in the mountains above Pasadena (chronicled in Chris Adami's
chapter @add citation here). These close interactions not only fueled
his research for the remainder of the year, but also were a deep
personal friendship. It was clear that, to Gerry, there was little
distinction between personal and professional in his research life.

\section{Science is done by people, and people need to be supported.}

Gerry was incredibly proud of his mentoring record, and did his best
to support his students, postdocs, and junior colleagues both
professionally and personally. He devoted the weeks around Christmas
each year to writing recommendation letters for junior colleagues. He
spent years working to successfully nominate colleagues to the
National Academy of Sciences. He supported junior faculty with
significant amounts of his time and sometimes by forgoing his own
salary to boost theirs. While he never stated it explicitly, believe
that he considered most ideas somewhat ephemeral, and that his real
legacy --  and the legacy most worth having -- lay in the students and
colleagues who would continue after him.

\section{Always treat the administrative staff well.}

Gerry was fond of pointing out that the secretaries and administrative
staff had more practical power than most faculty, and that it was
worth staying on their good side. This was less a statement of
calculated intent and more an observation that many students,
postdocs, and faculty treated non-scientists with less respect than
they deserved. He always took the time to interact with them on a
personal level, and certainly seemed to be well liked for it. I've
been told by several colleagues who worked with Gerry that this was a
lesson that they took to heart in their own interactions with staff,
and it has also served me well.

\section{Hard work is more important than brilliance.}

One of Gerry's favorite quotes was 'Success is 99% perspiration, 1%
inspiration', a statement attributed to Thomas Edison. According to
Gerry, he simply wasn't as smart as many of his colleagues, but he
made up for it by working very hard. I have no idea how modest he was
being --  he was not always known for modesty -- but he certainly worked
very hard, spending 10-14 hours a day writing in his home office,
thinking in the garden, or at work. While I try to work fewer hours
myself, it has always been clear to me that sometimes hard work is a
necessary when tackling tricky problems: for example, the Earthshine
publications (@add citation here) came after a tremendously unpleasant
summer working on some incredibly messy analysis code, but without the
resulting analysis we wouldn't have been able to continue the project.

\section{Experiments should talk to theory, and vice versa.}

Steve Koonin once explained to me that Gerry was a phenomenologist -- a
theorist who worked well with experimental data -- and that this
specialty was fairly rare because it required communicating
effectively across sub disciplines. Gerry wasn't attracted to deep
theoretical work and complex calculations, and in any case liked to
talk to experimentalists too much to be a good theorist -- for example,
some of our most frequent dinner guests when I was growing up were
Peter Braun- Munzinger and Johanna Stachel, both experimentalists. So
he chose to work at the interface of theory and experiment, where he
could develop and refine his intuition based on competing world views
emanating from the theorists (who sought clean mathematical solutions)
and experimentalists (who had real data that needed to be reconciled
with theory). I have tried to pursue a similar strategy in
computational biology.

\section{Computers and mathematical models are tools, but the real insight
comes from intuition.}

Apart from some early experience with punch cards at Yale in the
1950s, Gerry avoided computers and computational models completely in
his own research (although his students, postdocs and collaborators
used them, of course). I am told that his theoretical models were
often relatively simple approximations, and he himself often said that
his work with Hans Bethe proceeded by choosing the right approximation
for the problem at hand -- something at which Bethe excelled. Their
choice of approximation was guided by intuition about the physical
nature of the problem as much as by mathematical insight, and they
could often use a few lines of the right equations to reach similar
results to complex computational and mathematical models. This search
for simple models and the utility of physical intuition in his
research characterized many of our conversations, even when I became
more mathematically trained.

\section{Teaching is largely about conveying intuition.}

Once a year, Gerry would load up a backpack with mason jars full of
thousands of pennies, and bring them into his Statistical Mechanics
class. This was needed for one of his favorite exercises ' a hands-on
demonstration of the law of large numbers and the Central Limit
Theorem, which lie at the heart of thermodynamics and statistical
mechanics. He would have students flip 100 coins and record the
average, and then do it again and again, and have the class plot the
distributions of results. The feedback he got was that this was a very
good way of viscerally communicating the basics of statistical
mechanics to students, because it built their intuition about how
averages really worked. This approach has carried through to my own
teaching and training efforts.

\section{Benign neglect is a good default for mentoring.}

Gerry was perhaps overly fond of the concept of 'benign neglect' in
parenting, in that much of my early upbringing was at the hands of my
mother with only occasional input from him. However, in his oft-stated
experience (and now mine as well), leaving smart graduate students and
postdocs to their own devices most of the time was far better than
trying to actively manage (or interfere in) their research for them. I
think of it this way: if I tell my students what to do and I'm wrong
(which is likely, research being research), then they either do it
(and I suffer for having misdirected them) or they don't do it (and
then I get upset at them for ignoring me). But if I don't tell my
students what to do, then they usually figure out something better for
themselves, or else get stuck and then come to me to discuss it. The
latter two outcomes are much better from a mentoring perspective than
the former two.

\section{Students need to figure it out for themselves.}

One of the most embarrassing (in retrospect) interactions I had with
my father was during a long car ride where he tried to convince me
that when x was a negative number, -x was positive. At the time, I
didn't agree with this at all, which was probably because I was only 7
or 8; I was also rather stubborn. While it took me a few more years to
understand this concept, by the time I was a math major I did have the
concept down; but in this, and many other interactions around science,
he never browbeat me about it or got upset at my stupidity or
stubbornness. I believe this carried through to his interactions with
his students. In fact, the only time I ever heard him express
exasperation was with colleagues who were acting badly.

\section{A small nudge at the right moment is sometimes all that is needed.}

A pivotal moment in my life came when Gerry introduced me to Mark
Galassi, a physics graduate student who also was the systems
administrator for the UNIX systems in the Institute for Theoretical
Physics at Stony Brook; Mark found out I was interested in computers
and gave me access to the compute system. This was one of the defining
moments in my research life, where I now work almost exclusively with
computation! Similarly, when I decided to take a year off from
college, my father put me in touch with Steve Koonin, who was looking
for a systems administrator for a new project; I ended up working with
the Earthshine project, which was a core part of my research for
several years. And when I was trying to decide what grad schools to
apply to, Gerry suggested I ask Hans Bethe and Steve Koonin what they
thought was a promising area of research -- their unequivocal answer
was ``biology!''  This led to me applying to biology graduate schools, and
ultimately to my current faculty position. In all these cases, I now
recognize the application of a light touch at the right moment, rather
than the heavy-handed guidance that he must have wanted to give at
times.

-----

There are many more personal stories that could be told about Gerry
Brown, including his (many, and hilarious) interactions with the East
German secret police during the cold war, his (quite bad) jokes, his
(quite good) cooking, and his (enthusiastic) ballroom dancing, but I
will save those for another time. I hope that his friends and
colleagues will see him in the examples above, and will remember him
fondly.


\subsection*{Acknowledgments}

I thank Chris Adami, Erich Schwarz, Tracy Teal, and my mother,
Elizabeth Brown, for their comments on drafts of this article.

\end{document}
